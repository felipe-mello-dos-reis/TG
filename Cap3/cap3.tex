\section{Materiais e caracterização}
\label{sec:materiais_e_caracterizacao}

Para o desenvolvimento das argamassas geopoliméricas monocomponentes, utilizarram-se os seguintes compontentes:

\begin{itemize}
    \item Metacaulim (MK) fornecido pela empresa \textcolor{red}{XXX};
    \item Sílica ativa (SF) fornecida pela empresa \textcolor{red}{XXX};
    \item Carbonato de potássio sólido da empresa \textcolor{red}{XXX};
    \item Hidróxido de cálcio da empresa \textcolor{red}{XXX};
    \item Areia padronizada de quartzo da empresa \textcolor{red}{XXX};
    \item Água destilada.
\end{itemize}

Os aglomerantes utilizados foram a sílica ativa e o metacaulim. Entretanto, a pureza encontrada no mercado do metacaulim não é suficiente para garantir precisão na caracterização das amostras cimentíceas.
Portanto, ele foi produzido a partir do caulim comercial, conforme detalhado na Seção \ref{sec:producao_do_metacaulim}.
Já as fontes alcalinas são encontradas no mercado com alta pureza, portanto foi utilizado a composição fisico-quimica forncedida pelo fabricante.

A composiçãocão química dos materiais empregados na formulação das argamassas é apresentada na Tabela \ref{tab:composicao_quimica_reagentes}.

\begin{table}[H]
    \caption{Propriedades químicas dos reagentes sólidos.}
    \label{tab:composicao_quimica_reagentes}
    \center
    \begin{tabular}{ccc}
    % after \\: \hline or \cline{col1-col2} \cline{col3-col4} ...
        \hline
        Material & Composição química & Especificação (\%)\\
        \hline
        Sílica ativa & $SiO_2$ &  \textcolor{red}{XX,X} \\
            & $ Al_2O_3$ & \textcolor{red}{XX,X} \\
            & $MgO$ & \textcolor{red}{XX,X} \\
            & $CaO$ & \textcolor{red}{XX,X} \\
            & $Fe_2O_3$ & \textcolor{red}{XX,X} \\
        Carbonato de potássio & $K_2CO_3$ & \textcolor{red}{XX,X} \\
        Hidróxido de cálcio & $Ca(OH)_2$ & \textcolor{red}{XX,X} \\
        \hline
    \end{tabular}
\end{table}

Além disso, a areia de quartzo utilizada segue os padrões estabelecidos pelo Instituto de Pesquisas Tecnológias (IPT), representados na Tabelas \ref{tab:areia_quartzo_propriedades} e \ref{tab:areia_quartzo_granulometria}.

\begin{table}[H]
    \caption{Resultados de requisitos fisicos e químicos da areia de quartzo padronizada.}
    \label{tab:areia_quartzo_propriedades}
    \center
    \begin{tabular}{p{0.25\textwidth} p{0.25\textwidth} p{0.25\textwidth}}
        \hline
        Propriedade & Resultado & Requisito ABNT NBR 7214:2015 \\
        \hline
        Teor de sílica (ABNT NBR 14656:2001) & 96,5\% & $\geq$ 95\%, em massa \\
        Umidade (ABNT NBR 7214:2015) & 0,0\% & $\leq$ 0,2\%, em massa \\
        Matéria orgânica (ABNT NBR 17053:2022) & Mais clara ou igual à cor da solução padrão & Cor da solução padrão de ácido tânico a 2\% \\
        \hline
    \end{tabular}
\end{table}

\begin{table}[H]
    \caption{Composição granulométrica das frações da areia de quartzo padronizada.}
    \label{tab:areia_quartzo_granulometria}
    \centering
    \begin{tabular}{p{0.10\textwidth} p{0.30\textwidth} p{0.15\textwidth} p{0.20\textwidth}}
        \hline
        \multirow{2}{=}{Fração} & \multirow{2}{=}{Intervalo entre peneiras} & \multicolumn{2}{c}{Porcentagem em massa (\%)} \\ \cline{3-4}       
        & & Resultado & Requisito ABNT NBR 7214:2015 \\
        \hline
        16 & (2,4 mm e 2,0 mm) & 0 & $\leq$ 10 \\
        16 & (2,0 mm e 1,2 mm) & 97 & $\geq$ 90 \\
        30 & (1,2 mm e 0,6 mm) & 99 & $\geq$ 95 \\
        50 & (0,6 mm e 0,3 mm) & 96 & $\geq$ 95 \\
        100 & (0,3 mm e 0,15 mm) & 95 & $\geq$ 95 \\
        \hline
    \end{tabular}
\end{table}

\section{Produção do metacaulim}
\label{sec:producao_do_metacaulim}

O metacaulim foi obtido a partir da calcinação do caulim a $700^\circ$C por 1 hora, em um forno elétrico da marca \textcolor{red}{XXX}.
O tempo de calcinação e temperatura ótimos foram determinados a partir de ensaios prévios, onde foi avaliado o rendimento da calcinação.
Para garantir a homogeinidade do material, foram utilizadas duas formas rasas com altura máxima de \textcolor{red}{XX} mm.
A transformação da caulinita cristalina em metacaulinita amorfa é representada pela Equação \ref{eq:calcinacao_caulim}.

\begin{equation}
    \label{eq:calcinacao_caulim}
        Al_2.2Si_2O_2.2H_2O \xrightarrow{\Delta} Al_2O_3.2SiO_2 + 2 H_2O
\end{equation}

\section{Caracterização físico-químico dos precursores sólidos}
\label{sec:caracterizacao_fisico_quimica_dos_precursores_solidos}

A caracterização físico-química dos precursores sólidos foi realizada nos laboratórios do Instituo Tecnológico de Aeronáutica (ITA), localizado em São José dos Campos-SP.

A espectroscopia de energia dispersiva de raios-X (EDS) permitiu determinar a proporção dos elementos quiímicos, realizada juntamente com a microscopia eletrônica de varredura (SEM) para avaliação da morfologia dos precursores sólidos.

Além disso, a difração de raios-X (XRD) foi utilizada para determinar a fase cristalina dos metacaulim e da sílica ativa.

% A determinação da composição química dos materiais precursores (metacaulim, escória de alto forno e sílica ativa) foi realizada no Centro de Desenvolvimento Técnico Nuclear (CDTN), localizado na Universidade Federal de Minas Gerais (UFMG), por meio do espectrômetro de energia dispersiva de raios X (EDS), modelo EDX-720 da Shimadzu, utilizando-se um método quantitativo. A composição mineralógica desses materiais foi obtida por meio de difração de raios X, empregando-se um difratômetro da marca Shimadzu, modelo XRD-7000, com radiação de cobre (Cu-Kα, λ = 1,5418 Å), operando a 40 kV e 30 mA. Para a determinação das fases cristalinas, foram realizadas varreduras com velocidade angular de 0,02° por segundo, dentro de um intervalo de medida entre os ângulos de Bragg (2θ) de 5° a 80°.

A fim de verificar a remoção dos grupos hidoxilas ($OH^-$) e a presença das ligações $Al-Si-O$, foi realizada a espectroscopia de infravermelho (FTIR) em um espectrômetro da marca \textcolor{red}{XXX}.

Por fim, a distribuição do tamanho de partícula dos sólidos empregados foi realizada através do ensaio de granulometria a laser. Partículas menores e mais irregulares tendem a apresentar maior área superficial específica e, portanto, maior reatividade em contato com a fonte alcalina, o que pode influenciar diretamente no desempenho mecânico e nas propriedades reológicas das argamassas geopoliméricas.

% As distribuições granulométricas do metacaulim (MC), escória de alto forno (EAF) e sílica ativa (SA) foram determinadas por difração a LASER, utilizando-se um equipamento CILAS1090. As amostras foram dispersas em água destilada, e as condições de ensaio adotadas incluíram agitação a 1500 rpm, tempo de ultrassom de 2,5 minutos, obscuração entre 10 e 20%, e tempo total de dispersão de 5 minutos.

