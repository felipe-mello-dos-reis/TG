\section{Contextualização}

Cimento é um dos principais materiais na construção civil, sendo utilizado desde a construção de residências e edifícios, até pontes e rodovias. Em países em desenvolvimento como o Brasil, o cimento é amplamente utilizado devido à sua baixa complexidade e custo, que permite seu uso em escala em qualquer local. O aumento exponencial da produção de cimento, 10 vezes maior que o crescimento populacional mundial \cite{united1995world} veio acompanhado de uma parcela expressiva da emissão de gases de efeito estufa (GHG), devido ao processo de calcinação do calcário que transforma o carbonato de cácio em óxido de cálcio de gás carbônico em fornos de alta temperatura. A produção do Cimento Portland gera em média 842 kg de $CO_2/t$ de clinker produzido \cite{andrew2018global}, representando 5\% das emissões antropogênicas de GHG \cite{IEA_WBCSD_2009}.

Neste contexto, surge a necessidade do desenvolvimento de novos materiais cimentícios que apresentem três propriedades principais: baixa emissão de GHG, baixo custo e alta resistência/durabildiade \cite{scrivener2018eco}.

\begin{figure}[ht]
\centering
\includegraphics[width=0.5\textwidth]{Cap1/cupim}
\caption{Proibido estacionar cupins. Legenda grande, com o objetivo de demonstrar a indentação na lista de figuras.}
\label{cupim}
\end{figure}