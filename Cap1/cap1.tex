O cimento é um dos principais materiais na construção civil, sendo utilizado desde a construção de residências e edifícios até pontes e rodovias.
Em países em desenvolvimento como o Brasil, o cimento é amplamente utilizado devido à sua baixa complexidade e custo, o que permite seu uso em escala em qualquer localidade.
O aumento exponencial da produção de cimento, 10 vezes maior que o crescimento populacional mundial \cite{united1995world}, veio acompanhado de uma parcela expressiva da emissão de gases de efeito estufa (GHG), devido ao processo de calcinação do calcário que transforma o carbonato de cálcio em óxido de cálcio e gás carbônico em fornos de alta temperatura.
A produção do Cimento Portland gera em média 842 kg de $CO_2/t$ de clínquer produzido \cite{andrew2018global}, representando 5\% das emissões antropogênicas de GHG \cite{IEA_WBCSD_2009}.

Neste contexto, surge a necessidade do desenvolvimento de novos materiais cimentícios que apresentem três propriedades principais: baixa emissão de GHG, baixo custo e alta resistência/durabilidade \cite{scrivener2018eco}.

Nos últimos anos, os materiais ativados alcalinamente (AAM) - precursores sólidos ricos em sílica ($SiO_2$) e alumina ($Al_2O_3$), capazes de formarem géis aglomerantes constituídos de sódio-alumino-silicato hidratado (NASH) e cálcio-alumino-silicato hidratado (CASH) - ganharam destaque devido ao seu potencial para substituir parcial ou totalmente o cimento Portland, reduzindo significativamente as emissões de GHG associadas à produção de cimento convencional.

Existem duas maneiras pelas quais os AAM podem ser produzidos: misturando o precursor sólido com um ativador alcalino líquido, ou com uma fonte alcalina sólida e água.
Os sistemas de duas partes foram amplamente empregados no desenvolvimento inicial deste mercado devido ao elevado desempenho mecânico, durabilidade e resistência química.
No entanto, os sistemas de uma parte são uma tecnologia mais escalável devido ao menor risco de manuseio e armazenamento dos ativadores sólidos \cite{provis2018alkali}.

Os precursores sólidos ricos em cálcio são mais utilizados por diversos fatores, como o rápido ganho de resistência \cite{provis2014geopolymers}, menor dependência de cura térmica \cite{ke2021one}, além de que os produtos da reação CASH tendem a formar uma matriz mais densa e menos porosa que os géis de NASH \cite{bernal2014engineering}.
Neste sentido, existe uma lacuna técnica e científica na formulação e caracterização de geopolímeros monocomponentes de baixo teor de cálcio.

O presente trabalho propõe o desenvolvimento de um cimento ativado alcalinamente monocomponente com foco em precursores sólidos de baixo teor de cálcio, especificamente metacaulim e sílica ativa, combinados a fontes alcalinas alternativas mais seguras e acessíveis, como carbonato de potássio ($K_2CO_3$) e hidróxido de cálcio ($Ca(OH)_2$).
Essa abordagem visa contribuir para a formulação de ligantes mais sustentáveis, seguros e com desempenho adequado para aplicações na construção civil, alinhando-se às diretrizes contemporâneas de construção de baixo impacto ambiental.