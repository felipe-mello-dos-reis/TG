\section{Materiais e caracterização}

Para o desenvolvimento das argamassas geopoliméricas monocomponentes, utilizarram-se os seguintes compontentes:

\begin{itemize}
    \item Metacaulim (MK) fornecido pela empresa \textcolor{red}{XXX};
    \item Sílica ativa (SF) fornecida pela empresa \textcolor{red}{XXX};
    \item Carbonato de potássio sólido da empresa \textcolor{red}{XXX};
    \item Hidróxido de cálcio da empresa \textcolor{red}{XXX};
    \item Areia padronizada de quartzo da empresa \textcolor{red}{XXX};
    \item Água destilada.
\end{itemize}

O principal aglomerante utilizado foi o metacaulim, que é um precursor sólido rico em sílica e alumina, entretanto, sua pureza encontrada no mercado não é suficiente para garantir precisão na caracterização das amostras cimentíceas.
Portanto, o metacaulim foi produzido a partir do caulim comercial.
Já os demais reagentes sólidos são encontrados no mercado com alta pureza, portanto foi utilizado a composição fisico-quimica forncedida pelo fabricante.

A composiçãocão química dos materiais empregados na formulação das argamassas é apresentada na Tabela \ref{tab:composicao_quimica_reagentes}.

\begin{table}[htb]
    \caption{Propriedades químicas dos reagentes sólidos.}
    \label{tab:composicao_quimica_reagentes}
    \center
    \begin{tabular}{ccc}
    % after \\: \hline or \cline{col1-col2} \cline{col3-col4} ...
        \hline
        Material & Composição química & Especificação (\%)\\
        \hline
        Sílica ativa & $SiO_2$ &  \textcolor{red}{XX,X} \\
            & $ Al_2O_3$ & \textcolor{red}{XX,X} \\
            & $MgO$ & \textcolor{red}{XX,X} \\
            & $CaO$ & \textcolor{red}{XX,X} \\
            & $Fe_2O_3$ & \textcolor{red}{XX,X} \\
        Carbonato de potássio & $K_2CO_3$ & \textcolor{red}{XX,X} \\
        Hidróxido de cálcio & $Ca(OH)_2$ & \textcolor{red}{XX,X} \\
        \hline
    \end{tabular}
\end{table}

Além disso, a areia de quartzo utilizada segue os padrões estabelecidos pelo Instituto de Pesquisas Tecnológias (IPT), representados na Tabelas \ref{tab:areia_quartzo_propriedades} e \ref{tab:areia_quartzo_granulometria}.

\begin{table}[htb]
    \caption{Resultados de requisitos fisicos e químicos da areia de quartzo padronizada.}
    \label{tab:areia_quartzo_propriedades}
    \center
    \begin{tabular}{p{0.25\textwidth} p{0.25\textwidth} p{0.25\textwidth}}
        \hline
        Propriedade & Resultado & Requisito ABNT NBR 7214:2015 \\
        \hline
        Teor de sílica (ABNT NBR 14656:2001) & 96,5\% & $\geq$ 95\%, em massa \\
        Umidade (ABNT NBR 7214:2015) & 0,0\% & $\leq$ 0,2\%, em massa \\
        Matéria orgânica (ABNT NBR 17053:2022) & Mais clara ou igual à cor da solução padrão & Cor da solução padrão de ácido tânico a 2\% \\
        \hline
    \end{tabular}
\end{table}

\begin{table}[htb]
    \caption{Composição granulométrica das frações da areia de quartzo padronizada.}
    \label{tab:areia_quartzo_granulometria}
    \centering
    \begin{tabular}{p{0.10\textwidth} p{0.30\textwidth} p{0.15\textwidth} p{0.20\textwidth}}
        \hline
        \multirow{2}{=}{Fração} & \multirow{2}{=}{Intervalo entre peneiras} & \multicolumn{2}{c}{Porcentagem em massa (\%)} \\ \cline{3-4}       
        & & Resultado & Requisito ABNT NBR 7214:2015 \\
        \hline
        16 & (2,4 mm e 2,0 mm) & 0 & $\leq$ 10 \\
        16 & (2,0 mm e 1,2 mm) & 97 & $\geq$ 90 \\
        30 & (1,2 mm e 0,6 mm) & 99 & $\geq$ 95 \\
        50 & (0,6 mm e 0,3 mm) & 96 & $\geq$ 95 \\
        100 & (0,3 mm e 0,15 mm) & 95 & $\geq$ 95 \\
        \hline
    \end{tabular}
\end{table}
