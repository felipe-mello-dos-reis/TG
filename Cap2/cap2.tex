\section{Contexto Histórico}

De acordo com \cite{pachecotorgal2014handbook}, a síntese de materiais através de ativação alcalina iniciou-se nas décadas de 1930 e 1940, quando foi sintetizado um substituto para o Cimento Portland tradicional a partir de escórias de alto-forno e outros aluminosilicatos.
Foi então a partir de 1970 que o interesse por esta área aumentou, quando o cientista francês Joseph Davidovits cunhou o termo "geopolímero" e patentetou diversas formulações de geopolímeros. Seu estudo inicial era voltado ao desenvilvimento de materiais inorgânicos não infamáveis e resistentes ao fogo \cite{provis2009geopolymers}.

Desde então, os AAM chamaram atenção de pesquisadores e indústrias devido ao baixo consumo de energia e natureza sustentável \cite{qin2022onepart}.
Além disso, conforme os estudos avançaram, os AAM ganharam reconhecimento por suas propriedades mecânicas e durabilidade, uma vez que as reações de polimerização que ocorrem durante a cura fornecem um alto ganho de resitência a compressão e ataques químicos.

\section{Matérias-primas de AAM}

\subsection{Precursores}

Os precursores são materiais ricos em $SiO_2$ e $Al_2O_3$ que, quando ativados por uma solução alcalina, formam uma rede tridimensional de aluminosilicatos \cite{rakhimova2019metakaolin}.
As propriedades mecânicas e cinéticas dos AAM são fortemente influenciadas pela proporção de $SiO_2/Al_2O_3$ \cite{provis2007geopolymerisation}.
O processo inicial da ativação consiste da dissolução dos aluminosilicatos pela quebra das ligações covalentes $Si-O-Si$ e $Al-O-Al$ quando o pH é elevado \cite{Severo2013}. A hidrólise pode ser esquematizada da seguinte forma:

\begin{equation}
  Al_2O_3 + 3H_2O + 2OH^- \rightarrow 2\left[Al(OH)_4\right]^- 
\end{equation}

\begin{equation}
  SiO_2 + H_2O + OH^- \rightarrow \left[SiO(OH)_3\right]^- 
\end{equation}

\begin{equation}
  SiO_2 + 2OH^- \rightarrow \left[SiO_2(OH)_2\right]^{2-}
\end{equation}

Em seguida, os silicatos e aluminatos dissolvidos reagem entre si para formar um gel, o qual passa por um processo de polimerização e endurecimento, conforme a Figura \ref{fig:ativacao}.

\begin{figure}[ht]
  \centering
  \includegraphics[width=0.75\textwidth]{Cap2/ativacao.png}
  \caption{Esquema do processo de ativação alcalina \cite{duxson2006geopolymer}.}
  \label{fig:ativacao}
\end{figure}

Os precursores podem ser divididos em duas categorias, os de alto teor de cálcio, como escória de alto-forno e cinza volante, e os de baixo teor de cálcio, como o metacaulim.
A Figura \ref{fig:diagrama_ternario} apresenta os precursores mais comuns e suas respectivas composições químicas.
\begin{figure}[ht]
  \centering
  \includegraphics[width=0.625\textwidth]{Cap2/diagrama_ternario.png}
  \caption{Diagrama ternário dos precursores mais comuns \cite{giergiczny2019fly}.}
  \label{fig:diagrama_ternario}
\end{figure}

O primeiro grupo possui como principal produto da reação de ativação o silicato de aluminio e cálcio hidratado (C-A-S-H), enquanto, o segundo grupo forma principalmente o gel de silicato de alumínio hidratado (N-A-S-H).

Quando níveis de cálcio presentes nesses precursores são altos, o produto final é em um gel de cura rápida, alta resistência inicial. No entanto, são mais suscetíveis a retrações, fissuras e corrosão por ataque de cloro.
Já os sistemas de baixo cálcio formam uma rede amorfa de tetraedros que apresenta baixa permeabilidade e retração, melhor resistência ao fogo e estrutura menos porosa.
A proporção de $SiO_2/Al_2O_3$ é responsável pelo grau de polimerizacão do gel formado, portanto, caso a relação ideal não seja atingida, a resistência mecânica e durabilidade do AAM podem ser comprometidas.
Por fim, o gel N-A-S-H necessita de um maior tempo de cura e temperatura entre $80-100\ ^\circ C$ para atingir a resistência mecânica apropriada \cite{Nodehi2021}.

A Tabela \ref{tab:principais_precursores} apresenta as principais características dos precursores mais comuns.

\begin{landscape}
\begin{table}[p]
  \centering
  \caption{Características de materiais residuais comuns ou inovadores que podem ser adicionados ao concreto para produzir aglomerantes mais sustentáveis \cite{Nodehi2021}.}
  \vspace{0.5cm}
  {\small % Reduzir o tamanho da fonte para a tabela
  \renewcommand{\arraystretch}{1.2} % Aumentar o espaçamento entre linhas
  \begin{tabular}{p{3.5cm} p{3cm} p{2cm} p{2cm} p{4cm} p{6cm}}
    \hline
    Nome do aditivo & Forma usual & Densidade média (kg/m\textsuperscript{3}) & Tamanho médio partícula (\textmu m) & Limitações & Benefícios \\
    \hline
    Cimento Portland (OPC) & Irregular e angular & 1440 & 0,15--20 & -- & -- \\
    Sílica ativa & Esférica & 2200 & 0,1--0,5 & Reduz trabalhabilidade e resistência inicial & Aumenta a compacidade, resistência mecânica e durabilidade \\
    Escória de alto-forno moída (GGBFS) & Angular com superfície rugosa & 1000--1300 & 1,25--250 & Baixa resistência inicial & Aumenta a durabilidade, melhora a ITZ e resistência a sulfatos \\
    Cinza volante (Fly ash) & Esférica & 540--860 & 0,5--300 & Baixa resistência inicial & Melhora a trabalhabilidade e a resistência a longo prazo \\
    Metacaulim & Porosa, lamelar e angular & 890 & 1--20 & Reduz a trabalhabilidade & Preenche a microestrutura e melhora a ITZ \\
    Cinza de casca de arroz & Irregular com alta porosidade & 504--700 & 5--10 & Variação nas propriedades e baixa reatividade & Alto teor de sílica; melhora a compacidade e resistência \\
    Pó de vidro & Irregular & 2500 & 0,8--50 & Alta contaminação & Melhora a durabilidade e reação pozolânica \\
    Lama vermelha (red mud) & Irregular e em forma de agulha & 2700--3400 & 100 até mais de 200 & Alta contaminação & Alto teor de alumina, pode melhorar a hidratação \\
    Resíduos cerâmicos & Angular & ~1700 & Abaixo de 100 & -- & Melhora a compacidade e o desempenho \\
    Escória de incineração de resíduos sólidos urbanos (MSWI) & Irregular & 660--1690 & -- & -- & Melhora a microestrutura e reduz a porosidade \\
    Cinza de lodo de papel & Irregular & Abaixo de 100 & -- & -- & Ajusta favoravelmente a razão S/A \\
    \hline
  \end{tabular}
  }
\label{tab:principais_precursores}
\end{table}
\end{landscape}

% \subsubsection{Metacaulim}

% \subsubsection{Sílica ativa}

\subsection{Ativadores}

O ataque alcalino sobre a microestrutura dos precursores resulta na liberação de silicatos e aluminatos na solução.
A solubilidade da sílica e da alumina em função do pH é apresentada na Figura \ref{fig:solubilidade}.

\begin{figure}[ht]
  \centering
  \includegraphics[width=0.625\textwidth]{Cap2/solubilidade.png}
  \caption{Solubilidade da sílica e da alumina em função do pH \cite{mason1952principles}.}
  \label{fig:solubilidade}
\end{figure}

Observa-se que a solubilidade da sílica é baixa em ambientes ácidos e alta em básicos, enquanto alumina é solúvel em ambos os extremos.
Portanto, para que as reações de ativação ocorram, é necessário que o pH da solução seja elevado.

Os ativadores alcalinos podem ser encontrados em duas formas, líquida - produzindo os geopolímeros de duas partes -, ou sólida - geopolímeros uma parte.
Os principais ativadores alcalinos líquidos são: hidróxido de sódio ($NaOH$), silicato de sódio ($Na_2SiO_3$), hidróxido de potássio ($KOH$), carbonato de sódio ($Na_2CO_3$), carbonato de sódio ($K_2CO_3$) e hidróxido de potássio ($KO$).
Os primeiros estudos de AAM focaram em ativadores líquidos, uma vez que o produto final apresenta alta resistência a compressão, aderência e capacidade de surportar cargas de fadiga. Além disso, também demonstram elevada resistência, ciclos de congelamento e descongelamento e altas temperaturas \cite{heath2014gwp}.

Apesar das vantagens dos sistemas de duas partes, as soluções básicas são corrosivas e irritam a pele humana, tornando o seu transporte e manuseio perigoso para os trabalhadores \cite{awoyera2019critical}.
Outro ponto que merece a atenção é a produção de silicato de sódio acontece entre $1200-1400\ ^\circ C$ e emite aproximadamente 1,514 kg de $CO_2$ por kg de silicato produzido, além de contribuir significativamente para a poluição do ar por meio de poeira e óxidos de nitrogênio e enxofre \cite{rajan2020sustainable}.

Dessa forma, os sistemas de uma parte surgem como uma alternativa mais segura, uma vez que os ativadores sólidos são menos perigosos e mais fáceis de manusear. Mesmo que os geopolímeros de uma parte apresentem menor resistência mecânica e necessitem cura térmica para atingir o desempenho adequado \cite{provis2018alkali}, seu uso é mais escalável.

Estima-se que os impactos ambientais dos ativadores sólidos e líquidos sejam de 24\% e 60\% do impacto causado pelo cimento Portland tradicional, respectivamente \cite{luukkonen2017review}.