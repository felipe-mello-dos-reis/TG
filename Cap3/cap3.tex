\section{Materiais e caracterização}
\label{sec:materiais_e_caracterizacao}

Para o desenvolvimento das argamassas geopoliméricas monocomponentes, utilizarram-se os seguintes compontentes:

\begin{itemize}
    \item Metacaulim (MK) fornecido pela empresa \textcolor{red}{XXX};
    \item Sílica ativa (SF) fornecida pela empresa \textcolor{red}{XXX};
    \item Carbonato de potássio sólido da empresa \textcolor{red}{XXX};
    \item Hidróxido de cálcio da empresa \textcolor{red}{XXX};
    \item Areia padronizada de quartzo da empresa \textcolor{red}{XXX};
    \item Água destilada.
\end{itemize}

Os aglomerantes utilizados foram a sílica ativa e o metacaulim. Entretanto, a pureza encontrada no mercado do metacaulim não é suficiente para garantir precisão na caracterização das amostras cimentíceas.
Portanto, ele foi produzido a partir do caulim comercial, conforme detalhado na Seção \ref{sec:producao_do_metacaulim}.
Já as fontes alcalinas são encontradas no mercado com alta pureza, portanto foi utilizado a composição fisico-quimica forncedida pelo fabricante.

A composiçãocão química dos materiais empregados na formulação das argamassas é apresentada na Tabela \ref{tab:composicao_quimica_reagentes}.

\begin{table}[H]
    \caption{Propriedades químicas dos reagentes sólidos.}
    \label{tab:composicao_quimica_reagentes}
    \center
    \begin{tabular}{ccc}
    % after \\: \hline or \cline{col1-col2} \cline{col3-col4} ...
        \hline
        Material & Composição química & Especificação (\%)\\
        \hline
        Sílica ativa & $SiO_2$ &  \textcolor{red}{XX,X} \\
            & $ Al_2O_3$ & \textcolor{red}{XX,X} \\
            & $MgO$ & \textcolor{red}{XX,X} \\
            & $CaO$ & \textcolor{red}{XX,X} \\
            & $Fe_2O_3$ & \textcolor{red}{XX,X} \\
        Carbonato de potássio & $K_2CO_3$ & \textcolor{red}{XX,X} \\
        Hidróxido de cálcio & $Ca(OH)_2$ & \textcolor{red}{XX,X} \\
        \hline
    \end{tabular}
\end{table}

Além disso, a areia de quartzo utilizada segue os padrões estabelecidos pelo Instituto de Pesquisas Tecnológias (IPT), representados na Tabelas \ref{tab:areia_quartzo_propriedades} e \ref{tab:areia_quartzo_granulometria}.

\begin{table}[H]
    \caption{Resultados de requisitos fisicos e químicos da areia de quartzo padronizada.}
    \label{tab:areia_quartzo_propriedades}
    \center
    \begin{tabular}{p{0.25\textwidth} p{0.25\textwidth} p{0.25\textwidth}}
        \hline
        Propriedade & Resultado & Requisito ABNT NBR 7214:2015 \\
        \hline
        Teor de sílica (ABNT NBR 14656:2001) & 96,5\% & $\geq$ 95\%, em massa \\
        Umidade (ABNT NBR 7214:2015) & 0,0\% & $\leq$ 0,2\%, em massa \\
        Matéria orgânica (ABNT NBR 17053:2022) & Mais clara ou igual à cor da solução padrão & Cor da solução padrão de ácido tânico a 2\% \\
        \hline
    \end{tabular}
\end{table}

\begin{table}[H]
    \caption{Composição granulométrica das frações da areia de quartzo padronizada.}
    \label{tab:areia_quartzo_granulometria}
    \centering
    \begin{tabular}{p{0.10\textwidth} p{0.30\textwidth} p{0.15\textwidth} p{0.20\textwidth}}
        \hline
        \multirow{2}{=}{Fração} & \multirow{2}{=}{Intervalo entre peneiras} & \multicolumn{2}{c}{Porcentagem em massa (\%)} \\ \cline{3-4}       
        & & Resultado & Requisito ABNT NBR 7214:2015 \\
        \hline
        16 & (2,4 mm e 2,0 mm) & 0 & $\leq$ 10 \\
        16 & (2,0 mm e 1,2 mm) & 97 & $\geq$ 90 \\
        30 & (1,2 mm e 0,6 mm) & 99 & $\geq$ 95 \\
        50 & (0,6 mm e 0,3 mm) & 96 & $\geq$ 95 \\
        100 & (0,3 mm e 0,15 mm) & 95 & $\geq$ 95 \\
        \hline
    \end{tabular}
\end{table}

\section{Produção do metacaulim}
\label{sec:producao_do_metacaulim}

O metacaulim foi obtido a partir da calcinação do caulim a $700^\circ$C por 1 hora, em um forno elétrico da marca \textcolor{red}{XXX}.
O tempo de calcinação e temperatura ótimos foram determinados a partir de ensaios prévios, onde foi avaliado o rendimento da calcinação.
Para garantir a homogeinidade do material, foram utilizadas duas formas rasas com altura máxima de \textcolor{red}{XX} mm.
A transformação da caulinita cristalina em metacaulinita amorfa é representada pela Equação \ref{eq:calcinacao_caulim}.

\begin{equation}
    \label{eq:calcinacao_caulim}
        Al_2.2Si_2O_2.2H_2O \xrightarrow{\Delta} Al_2O_3.2SiO_2 + 2 H_2O
\end{equation}

\section{Caracterização físico-químico dos precursores sólidos}
\label{sec:caracterizacao_fisico_quimica_dos_precursores_solidos}

A caracterização físico-química dos precursores sólidos foi realizada nos laboratórios do Instituo Tecnológico de Aeronáutica (ITA), localizado em São José dos Campos-SP.

A espectroscopia de energia dispersiva de raios-X (EDS) permitiu determinar a proporção dos elementos quiímicos, realizada juntamente com a microscopia eletrônica de varredura (SEM) para avaliação da morfologia dos precursores sólidos.

Além disso, a difração de raios-X (XRD) foi utilizada para determinar a fase cristalina dos metacaulim e da sílica ativa.

% A determinação da composição química dos materiais precursores (metacaulim, escória de alto forno e sílica ativa) foi realizada no Centro de Desenvolvimento Técnico Nuclear (CDTN), localizado na Universidade Federal de Minas Gerais (UFMG), por meio do espectrômetro de energia dispersiva de raios X (EDS), modelo EDX-720 da Shimadzu, utilizando-se um método quantitativo. A composição mineralógica desses materiais foi obtida por meio de difração de raios X, empregando-se um difratômetro da marca Shimadzu, modelo XRD-7000, com radiação de cobre (Cu-Kα, λ = 1,5418 Å), operando a 40 kV e 30 mA. Para a determinação das fases cristalinas, foram realizadas varreduras com velocidade angular de 0,02° por segundo, dentro de um intervalo de medida entre os ângulos de Bragg (2θ) de 5° a 80°.

A fim de verificar a remoção dos grupos hidoxilas ($OH^-$) e a presença das ligações $Al-Si-O$, foi realizada a espectroscopia de infravermelho (FTIR) em um espectrômetro da marca \textcolor{red}{XXX}.

Por fim, a distribuição do tamanho de partícula dos sólidos empregados foi realizada através do ensaio de granulometria a laser. Partículas menores e mais irregulares tendem a apresentar maior área superficial específica e, portanto, maior reatividade em contato com a fonte alcalina, o que pode influenciar diretamente no desempenho mecânico e nas propriedades reológicas das argamassas geopoliméricas.

% As distribuições granulométricas do metacaulim (MC), escória de alto forno (EAF) e sílica ativa (SA) foram determinadas por difração a LASER, utilizando-se um equipamento CILAS1090. As amostras foram dispersas em água destilada, e as condições de ensaio adotadas incluíram agitação a 1500 rpm, tempo de ultrassom de 2,5 minutos, obscuração entre 10 e 20%, e tempo total de dispersão de 5 minutos.

\section{Produção das argamassas e pastas geopoliméricas}
\label{sec:producao_das_argamassas_e_pastas_geopolimericas}

\subsection{Formulação das misturas}
\label{subsec:formulacao_das_misturas}

O desenvolvimento das argamassas geopoliméricas monocomponentes seguiu um planejamento experimental sistemático, visando avaliar o efeito das diferentes composições nas propriedades fisico-químicos e mecânicas.

As variáveis consideradas no estudo foram:

\begin{itemize}
    \item Proporção entre os precursores sólidos (metacaulim e sílica ativa);
    \item Teor de ativadores alcalinos ($K_2CO_3$ e $Ca(OH)_2$);
    \item Relação água/sólidos (a/s);
    \item Relação areia/material cimentício (ar/c).
\end{itemize}

A variável de estudo deste experimento é a relação $Si/Al$, que será variada de 1,0 até 5,0, sendo a relação $Si/Al$ calculada com base na proporção de metacaulim e sílica ativa.

Inicialmente, a proporção de água/sólidos foi determinada de maneira análogo a \textcolor{red}{ref xxxx}, visando garantir a trabalhabilidade adequada das pastas.

Além disso, devido ao balanço estequiométrico, a relação $Al/K$ será constante e igual a 1, conforme a formula empírica \ref{eq:relacao_al_k} \cite{joseph1991geopolymers}, onde $M$ é um cátion de sódio ou potássio.

\begin{equation}
    \label{eq:relacao_al_k}
    M_n \left\{ \left(SiO_2 \right)_z AlO_2 \right\}_n \cdot wH_2O
\end{equation}

Ademais, a relação $K/Ca$ será constante e igual a 2, respeitando a reação de precipitação do carbonato de potássio com o hidróxido de cálcio, conforme a Equação \ref{eq:reacao_k_ca}.

\begin{equation}
    \label{eq:reacao_k_ca}
    K_2CO_3 + Ca(OH)_2 \rightarrow  2KOH_{(aq)} + CaCO_{3(s)} \downarrow
\end{equation}

Tendo as proporções da pasta bem definidas, a produção das argamassas manteve a proporção de 1:3 entre o aglomerante e a areia, conforme a literatura \textcolor{red}{ref XXXX}.

% For the chemical designation of geopolymers based on silico-aluminates, poly(sialate) was suggested. Sialateis an abreviation for silicon-oxo-aluminate. The sialate network consists of SiO4 and AlO4 tetrahedra linkedalternately by sharing all the oxygens. Positive ions (Na+, K+, Li+, Ca++, Ba++, NH4+, H3O+) must bepresent in the framework cavities to balance the negative charge of Al3++ in IV-fold coordination.Poly(sialates) have this empirical formula

A Tabela \ref{tab:composicoes_argamassas} apresenta as diferentes formulações produzidas, com as respectivas proporções em massa dos componentes.

\begin{table}[H]
    \caption{Composições das argamassas geopoliméricas produzidas.}
    \label{tab:composicoes_argamassas}
    \center
    \begin{tabular}{cccccccc}
    \hline
    Amostra & \multicolumn{2}{c}{Precursores (\%)} & \multicolumn{2}{c}{Ativadores (\%)} & \multirow{2}{*}{a/s} & \multirow{2}{*}{ar/c} & \multirow{2}{*}{Si/Al} \\
    \cline{2-5}
     & MK & SF & $K_2CO_3$ & $Ca(OH)_2$ & & \\
    \hline
     AGP1 & \textcolor{red}{XX} & \textcolor{red}{XX} & \textcolor{red}{X,X} & \textcolor{red}{X,X} & \textcolor{red}{X,XX} & \textcolor{red}{X,X} & \textcolor{red}{1,0} \\
    AGP2 & \textcolor{red}{XX} & \textcolor{red}{XX} & \textcolor{red}{X,X} & \textcolor{red}{X,X} & \textcolor{red}{X,XX} & \textcolor{red}{X,X} & \textcolor{red}{2,0} \\
    AGP3 & \textcolor{red}{XX} & \textcolor{red}{XX} & \textcolor{red}{X,X} & \textcolor{red}{X,X} & \textcolor{red}{X,XX} & \textcolor{red}{X,X} & \textcolor{red}{3,0} \\
    AGP4 & \textcolor{red}{XX} & \textcolor{red}{XX} & \textcolor{red}{X,X} & \textcolor{red}{X,X} & \textcolor{red}{X,XX} & \textcolor{red}{X,X} & \textcolor{red}{4,0} \\
    AGP5 & \textcolor{red}{XX} & \textcolor{red}{XX} & \textcolor{red}{X,X} & \textcolor{red}{X,X} & \textcolor{red}{X,XX} & \textcolor{red}{X,X} & \textcolor{red}{5,0} \\
    \hline
    \end{tabular}
\end{table}

\subsection{Procedimento de mistura}
\label{subsec:procedimento_de_mistura}

A produção das misturas seguiu os procedimentos normatizados.
Para a produção das argamassas e ensaio de compressão, segui-se os procedimentos da norma brasileira \cite{ABNT_NBR_7215_2019}, já para a produção das pastas, optou-se pela norma americana \cite{ASTM_C305_2006}, uma vez que a norma brasileira não especifica o procedimento de mistura para pastas cimentícias sem agregado miúdo.
Ambos os procedimentos foram adaptados para o preparo de amostras de volume reduzido.

\subsection{Moldagem e cura dos corpos de prova}
\label{subsec:moldagem_e_cura_dos_corpos_de_prova}

Para o ensaio de compressão, os corpos de prova foram preparados em moldes prismáticos de dimensões \textcolor{red}{XX} × \textcolor{red}{XX} × \textcolor{red}{XX} mm, previamente lubrificados com desmoldante à base de óleio.


Para cada composição, foram moldados 9 corpos de prova, destinados aos ensaios nas idades de 1, 3 e 7 dias (3 corpos de prova para cada idade). Não foi necessário realizar os testes com 28 dias pois a cura térmica dos aglomerantes empregados apresenta alto ganho de resistência inicial, conforme demonstrado na literatura \textcolor{red}{ref XXX}.

Optou-se por realizar a cura térmica em uma estufa mantida a (60 ± 2)°C e umidade relativa mínima de 95\% por 24 horas, conforme recomendado na norma \cite{ABNT9479ABNT_NBR_9479_2006}, para garantir a ativação dos aglomerantes e acelerar o processo de cura.
Vale ressaltar que o desmolde foi feito 24 horas após o início do processo de cura.

A desmoldagem foi realizada após 24 horas da moldagem, e os corpos de prova foram imediatamente transferidos para as condições de cura correspondentes até a idade de ensaio.

Para as análises microestruturais, pequenas amostras foram separadas, tendo a hidratação interrompida por imersão em alcool etílico e filtragem a vácuo, seguido de secagem em estufa a 40°C por 24 horas. Tais amostras foram armazenadas em recipientes herméticos para evitar a uma nova hidratação.

\section{Caracterização das misturas geopoliméricas}
\label{sec:caracterizacao_das_misturas_geopolimericas}

\subsection{Ensaios no estado endurecido}
\label{subsec:ensaios_no_estado_endurecido}

\subsubsection{Resistência à compressão}
\label{subsubsec:resistencia_a_compressao}

Para o ensaio de resistência à compressão, os corpos de provas foram inseridos em uma prensa hidráulica da marca \textcolor{red}{XXX}, modelo \textcolor{red}{XXX}, aplicando-se carga com velocidade de \textcolor{red}{XXX} N/s até a ruptura. A resistência foi calculada pela equação:

\begin{equation}
    \label{eq:resistencia_compressao}
    R_c = \frac{F_c}{A_t}
\end{equation}

Onde:
\begin{itemize}
    \item $R_c$ é a resistência à compressão, em MPa;
    \item $F_c$ é a carga máxima aplicada, em N;
    \item $A_t$ é a área da seção transversal, em mm\textsuperscript{2}.
\end{itemize}

Para análise estatística, realizou-se o teste de Tukey, permitindo identificar diferenças significativas entre os grupos de amostras, considerando um nível de significância de \textcolor{red}{XXX}\%.

\subsection{Análises microestruturais}
\label{subsec:analises_microestruturais}

\subsubsection{Difração de raios-X (XRD)}
\label{subsubsec:difracao_de_raios_x}

As análises de difração de raios-X foram realizadas em amostras moídas das pastas, com granulometria inferior a 75 µm, nas idades de 7 e 28 dias. Utilizou-se um difratômetro da marca \textcolor{red}{XXX}, modelo \textcolor{red}{XXX}.
%  com radiação de cobre (Cu-Kα, λ = 1,5418 Å), operando a 40 kV e 30 mA. As varreduras foram realizadas com velocidade angular de 0,02° por segundo, em um intervalo de 5° a 80° (2θ).

\subsubsection{Espectroscopia de infravermelho por transformada de Fourier (FTIR)}
\label{subsubsec:espectroscopia_de_infravermelho}

A espectroscopia de infravermelho foi realizada em amostras moídas das pastas, com granulometria inferior a 45 µm, nas idades de 1, 3 e 7 dias. Utilizou-se um espectrômetro da marca \textcolor{red}{XXX}, modelo \textcolor{red}{XXX}, na faixa de \textcolor{red}{XXX} a \textcolor{red}{XXX} cm$^{-1}$, com resolução de \textcolor{red}{XXX} cm$^{-1}$ e \textcolor{red}{XXX} varreduras. 

\subsubsection{Microscopia eletrônica de varredura (SEM)}
\label{subsubsec:microscopia_eletronica_de_varredura}

A análise microestrutural das pastas foi realizada por microscopia eletrônica de varredura, utilizando um microscópio da marca \textcolor{red}{XXX}, modelo \textcolor{red}{XXX}, acoplado a um espectrômetro de energia dispersiva de raios-X (EDS). As amostras foram preparadas a partir de fragmentos cilíndricos das pastas moldadas em um canudo de plástico descartável. 
% As análises foram realizadas com ampliações de 500×, 2000× e 5000×, com tensão de aceleração de 15 kV.

